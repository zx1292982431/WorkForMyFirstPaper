{第1章命题逻辑}
{内容提要}
{(1.命题符昊化及联结词}
命题与真值不是真就是假的陈述句称为命题。命题的判断结果称为命题的真值。真值只取两个值:真和假。真值为真的命题称为真命题,真值为假的命題称为假命题。由简单陒述句构成的命题称为简单命题或原子命题。命题符号化是用字母或带下角标的字母$p,q,r,\cdots,p_{i},q_{i},r_{i},\cdots$表示命题,用数字1表示真,用0表示假。由简单命题用联结词联结而成的命题称为复合命题。常用的联结词(逻辑联结词)及相关的复合命题有以下5种。
否定式设$p$为一个价题,复合命题“非$p$”(或“$p$的否定”)称为$p$的否定式,记作$\negp$。$\neg$为否定联结词。$\negp$为直当且仅当$p$为假。
合取式设$p、q$为两个命题,复合命題“$p$并且$q$”(或“$p$和$q$”)称为$p$与$q$的合取式,记作$p\wedgeq$。$\wedge$称为合取联结词。$p\wedgeq$为真当且仅当$p$与$q$同时为真。
析取式设$p、q$为两个命题,复合命題“$p$或$q$”称为$p$与$q$的析取式,记作$p\veeq$。V称为析取联结词。$p\veeq$为假当且仅当$p$与$q$同时为假。
蓝涵式设$p、q$为两个命题,复合命题“如果$p$,则$q$”为$p$与$q$的蕴涵式,记作$p\rightarrowq$,称$p$为蕴涵式的前件,$q$为䔽涵式的后件。一为蓝涵联结词。$p\rightarrowq$为假当且仅当$p$为真、$q$为假。
等价式设$p、q$为两个命題,复合命题“$p$当且仅当$q$”为$p$与$q$的等价式,记作$p\leftrightarrowq$。$\leftrightarrow$为等价联结词。$p\leftrightarrowq$为真当且仅当$p$与$q$的真值相同。
们题公式及分类
命题常项及命题变项若用$p,q,r,\cdots$表示确定的简单命题,则称$p,q,r,\cdots$为命题常项,命题常项的真值是确定不变的。若用$p,q,r,\cdots$表示真值可以变化的简单除述句,则称$p,q,r,\cdots$为命题变项,此时$p,q,r,\cdots$是变量,它们的取值为1或0。
合式公式
(1)单个的命题变项是合式公式。
(2)若$A$是合式公式,则$(\negA)$也是合式公式。
(4)只有有限次地应用(1)(3)形成的符号串才是合式公式。合式公式也称命题公式,简称公式。
对以上定义的说明如下。
(1)定义中的字母$A,B,\cdots$代表任意的公式。(2)联结词的优先顺序:ᄀ,$\wedge,\vee,\rightarrow,\leftrightarrow$。若有圆括号,先进行圆括号内的运算。相同的联结词按从压至在的顺序演第。
(3)公式的最外层圆括号有时可以省去,不改变运算顺序的圆括号也可省去。
公式的层次
(1)若$A$是单个的命题变项,则称$A$为0层公式。
(2)称$A$是$n+1(n\geqslant0)$层公式是指下列诸情况之一。
(3)若$A$的层肷为$k$,则称$A$为$k$层公式。
赋值或解释设$A$为一个公式,$p_{1},p_{2},\cdots,p_{n}$是出现在$A$中的全部命题变项,给$p_{1}$,$p_{2},\cdots,p_{n}$各指定一个真值$(0$或1)称为对$A$的一个珷值或解释。若赋值使$A$的真值为1,则称该赋值为$A$的成真颊值;若赋值使$A$的其值为0,则称该赋值为$A$的成假赋值。
真值表设公式$A$含$n(n\geqslant1)$个命題变项,将$A$在$2^{n}$个赋值下的取值情况列成表,称为$A$的具值表。
公式的分类设$A$为一个公式。
(1)若$A$无成假跔值,则称$A$为重至式或永真式。
(2)若$A$无成真赋值,则称$A$为矛盾式或永假式。
(3)若$A$至少有一个成真钱值,则称$A$为可满足式。
(4)若$A$至少有一个成真赋值,又至少有一个成假赋值,则称$A$为非重言式的可满足式。
{(3.)等们溽算}
等值式若等价式$A\leftrightarrowB$是重言式,则称$A$与$B$等值,记作$A\LeftrightarrowB$。基本的等值式
双重否定律
冥等律
交摸律
分配律
德摩根律
吸收律
零律
同一律
排中律
矛盾律
蕴涵等值式
等价等值式
假言易位
等价否定等值式
归谬论
等值演算由已知等值式推演出与给定公式等值的公式的过程称为等值演算。
文字命題变项及其否定统称为文字。
简单析取式由有限个文字组成的析取式称为简单析取式。
简单合取式由有限个文字组成的合取式称为简单合取式。
极小项设有$n$个命题变项,若在简单合取式中每个命题变项以文字的形式出现且仅出现一次,则称这样的简单合取式为极小项。$n$个命题变项共可产生$2^{n}$个不同的极小项,分别记为$m_{0},m_{1},\cdots,m_{2^{*}-1}$,其中,$i\left(0\leqslanti\leqslant2^{n}-1\right)$的二进制表示即为$m_{i}$的成真赋值。
极大项设有$n$个命题变项,若在简单析取式中每个命题变项以文字的形式出现且仅出现一次,称这样的简单析取式为极大项。$n$个命题变项共可产生$2^{n}$个不同的极大项,分别记为$M_{0},M_{1},\cdots,M_{2^{n-1}}$,其中,$i\left(0\leqslanti\leqslant2^{n}-1\right)$的二进制表示即为$M_{i}$的成假赋值。
在极小项和极大项中,文字通常按下角标或字典顺序排列。
析取范式由有限个简单合取式组成的析取式称为析取范式。
主析取范式由有限个极小项组成的析取范式称为主析取范式。
合取范式由有限个简单析取式组成的合取式称为合取范式。
主合取范式由有限个极大项组成的合取范式称为主合取范式。
{主要定理}
定理1.1任一命题公式都存在与其等值的析取范式和合取范式。
定理1.2任一命题公式都存在唯一的与其等值的主析取范式和主合取范式。
肤等润全功能集
联结词全功能集设$S$为一个联结词集合,若任意真值函数都可以用仅含$S$中的联结词的公式表示,则称$S$为联结词全功能集。
与非式设$p、q$为两个命題,复合命題“$p$与$q$的否定”称为$p$与$q$的与非式,记作$p\uparrowq$,即$p\uparrowq=\neg(p\wedgeq)$。$\uparrow$为与非联结调。$p\uparrowq$为假当且仅当$p$与$q$同时为真。
或非式设$p、q$为两个命题,复合命题“$p$或$q$的否定”称为$p$与$q$的或非式,记作$p\veeq$,即$p\downarrowq=\neg(p\veeq)$。$\downarrow$为或非联结词。$p\downarrowq$为真当且仅当$p$与$q$同时为假。
{6.细公电路}
设计组合电路的一般步胀如下。
(1)寻出问题的输人-输出表,即问题的真值函数。
(2)根据真值函数写出它的主析取范式。
(3)将主析取范式化简成最简恶开式,可采用亘因-英可拉斯基方法化简。
(7.)推㫜理论
推理的形式结构设$A_{1},A_{2},\cdots,A_{k},B$为命题公式,称
为推理的形式结构。$A_{1},A_{2},\cdots,A_{k}$为推理的前提,$B$为推理的结论。若(*)为重言式,则称推理正确,此时称$B$是$A_{1},A_{2},\cdots,A_{k}$的逻辑结论或存效结论,记为
推理定律称重言緼涵式为推理定律。主要的推理定律如下。
附加
化简
假言推理
拒取式
析取三段论
等价三段论
判断推理是否正确的方法判断推理是否正确,就是判断推理的形式结构(*)是否为重言式。其主要方法如下。
(1)真值表法。
(2)等值演算法。
(3)主析取(主合取)范式法。
构造证明法
证明证明是一个描述推理过程的命题公式序列,其中的每个命題公式或者为已知的前提,或者是由前面的公式应用推理规则得到的结论(中间结论)。
推理规则
(1)前提引人规则。
(2)结论引用规则。
(3)置换规则。
以下推理规则用图式给出,每个图式横线上面为前提,横线下面为结论。
(4)假言推理规则。
(5)附加规则。
(6)化简规则。
(7)拒取式规则。
(8)假言三段论规则。
(9)析取三段论规则。
(10)构造性二难规则。
(11)合取引人规则。
附加前提证朋法设推理的结论是薪涵式$A\rightarrowB$,把结论中的前件$A$作为前提,称为附加前提,证明结论中的后件$B$为有效结论。
归该法把推理的结论$B$的否定$\negB$作为前提,推出矛盾,即证明0为有效结论。
学习第1章(命题逻辑)要注意以下7点。
(1)要弄清命题与陈述句的关系。命题都是陈述句,但际述句不都是命题。只有陈述句所表达的判断结果是唯一确定的(正确的或错误的),它才是命題。
(2)并清由5种基本联结词联结的复合命题的逻辎关系及其真值。特别是要䒪清蕴涵式$p\rightarrowq$的逻辑关系及其真值。这里,$q$是$p$的必要条件。无论蕴涵关系如何表述,都要仔细地区分出缊涵式的剖件和后件,否则会将必要条件当成充分条件,当然就有可能将假命题变成真命题,或将直命题变成假命题。
(3)记住24个基本等值式,这是学好命题逻辑的关键。因为在等值演算过程中,在求主析取范式和主合取范式过程中,在将公式化成等值的某个全㘦能联结词集中公式的过程中都离不开基本等值式。
(4)要会准确地求出给定公式的主析取范式和主合取范式。掌握主析取范式与真值表及成真赋值的关系,主合取范式与真值表及成假珷值的关系,主析取范式与主合取范式的关䒺。恶清不同类型公式的主析取范式与主合取范式的特点。特别是要知道,重言式的主析取范式含$2^{n}(n$为公式中含的命题变项数)个极小项,主合取范式为1;而矛盾式的主析取范式为0,主合取范式含$2^{n}$个极大项。
(5)会用多种方法(如真值表法、等值演算法、主析取范式法等)判断公式的类型及判断两个公式是否等值。公式。
(6)会用等值演算法将一个联结词集上的公式等值地化为另一个联结词全功能集上的
(7)要弄清楚推理的形式结构,掌握判断推理是否正确的方法,对某些正确的推理会构造它的证明。
{第2章一阶圐辑}
{内容提要}
{(1.)一阶送铒基本概念}
个体词、谓词与量词在一阶逻辑中,简单命题被分解成主语和谓语两部分。表示主语的词(一般由名词或代词充当)称为个体词。具体或特定的个体词称为个体常项,抽象的或泛指的个体词称为个体变项,个体变项的取值范围称为个体域。由宇宙间一切事物组成的个体域称为全总个体域。表示语语的用来刻画个体词性质或个体闰之间关系的词称为谓词。谓词分为谓词常项和谓词变项。一般地,用$P\left(x_{1},x_{2},\cdots,x_{n}\right)$表示含$n(n\geqslant1)$个个体变项的$n$元谓词,它是以个体变项的个体域为定义域,以$\{0,1\}$为值域的$n$元函数。$n=1$时,$P(x)$表示$x$具有性质$P;n\geqslant2$时,$P\left(x_{1},x_{2},\cdots,x_{n}\right)$表示$x_{1},x_{2},\cdots,x_{n}$之间有关系$P$。为了讨论个体域中具有共同性质的个体的其他性质,首先要引进表示其共同性质的谓词,称这样的谓词为特性谓词。
表示数量的词称为量词。表示“存在”的量词称为存在量词,用了表示。表示“所有”的量词称为全称量词,用$\forall$表示。
{2.一阶䢍辑合式公式及其解释}
{字母表}
(7)圆括号与逗号:(,),,
项
(1)个体常项和个体变项是项;
(3)只有有限次地应用(1)、(2)生成的符号串才是项。
(1)原子公式是合式公式;
(2)若$A$为合式公式,则$(\negA)$也是合式公式;
(4)若$A$是合式公式,则$\forallxA、\existsxA$也是合式公式;
(5)只有有限次地应用(1)(4)生成的符号串才是合式公式,简称公式。
指导变元、辖域在公式$\forallxA$和$\existsxA$中,称$x$为指导变元,称$A$为相应量词的辖域。当$x$为指导变元时,$A$中$x$的所有出现都称为是约束出现,$A$中不是约束出现的个体变项称为自由出现。若在$\forallxA$和$\existsxA$中,无自由出现的个体变项,则称它们为闭式。
解释一个解释由4部分组成:
(1)非空个体域$D$;
(2)给论及的每个个体常项符号指定一个$D$中的元素;
(3)给论及的每个函数变项符号指定一个$D$上的函数;
(4)给论及的每个谓词变项符号指定一个$D$上的谓词。
赋值在给定的解释下,对公式中每个自由出现的个体变项指定个体域中的一个元素。在给定的解释$I$和赋值$\sigma$下,采用指定的个体域$D$,并将公式$A$中的所有个体常项符号、函数变项符号及谓词变项符号分别替换成$I$中指定的元素、函数及谓词,将$A$中所有自由出现的个体变项符号替换成$\sigma$指定的元素。
公式的分类若$A$在任何解释和该解释下的任何赋值下均为真,则称$A$为逻辑有效式或永真式;若$A$在任何解释和该解释下的任何賦值下均为假,则称$A$为矛盾式或永假式;若$A$至少存在一个成真的解释和该解释下的一个赋值,则称$A$为可满足式。
{主要定理}
定理2.1命题逻辑中重言式的代换实例都是逻辑有效式,命题逻辑中矛盾式的代换实例都是矛盾式。
{(3.)一阶送畩等值式与前果范式}
等值式设$A、B$为一阶逻辑公式,若$A\leftrightarrowB$为逻辑有效式,则称$A$与$B$等值,记作$A\LeftrightarrowB$。
前束范式若一阶逻辑公式$A$具有如下形式:
{主要定理}
定理2.2任何一阶逻辑公式都存在与之等值的前束范式(但形式不唯一)。
换名规则将一个指导变项及其在辖域中所有约束出现替换成公式中没有出现的个体变项符号。
通过使用换名规则得到的公式与原公式等值。
量词否定等值式
量词辖域收缩与扩张等值式
设公式$B$中不含$x$的自由出现。
量词分配等值式
消去量词
学习第2章(一阶逻辑)要注意以下几点。
(1)同一个命题在不同个体域内可能有不同的符号化形式,也可能有不同的真值,因而在将一个命題符号化之前,必须弄清个体域。若没有指定个体域,应采用全总个体域。
(2)在一阶逻辑命题符号化时,经常使用下面两种形式的公式:
其中,$F(x)、G(x)$为任意两个1元谓词,$F(x)$是特性谓词。
第一个公式的含义是“对于任意的个体$x$,如果$x$具有性质$F$,则$x$也有性质$G$”。第二个公式的含义是“存在个体$x$,具有性质$F$和性质$G$。”或者“存在具有性质$F$的个体$x$具有性质$G_{0}$”
注意不要把它们与下述两个公式混泽:
这两个公式的含义分别是“所有的个体$x$,都有性质$F$并且有性质$G$。”和“存在个体$x$,若$x$有性质$F$,则$x$有性质$G_{0}{}^{\prime\prime}$
(3)一阶逻辑公式共分3种类型;猡㮖有敦式(永真式)、予盾式(永假式)和可满足式。公式在任何解释和赋值下都是命题。对于闭式,只需要给定解㮫。
(4)记任主要的等值式,包括量词否定等值式、量词辎域收缩与扩张等值式、量词分配等值式、在有限个体域内消去量词。会用换名规则,会求结定公式的前束范式。
{第3章集合的基本概念和运算}
{内容提要}
{(1.集合与无素}
集合与元素是集合论的基本概念,联系元素和集合的是隶属关系。如果元素$x$属于集合$A$,则记作$x\inA$,否则记作$x\notinA$。
{2.集合与集合}
集合与集合之间的关系有包含$(\subseteq)$、相等$(=)$、不包含$(\nsubseteq)$、不相等$(\neq)$、真包含$(\subset)$、不真包含$(\not\subset)$等,具体定义如下:
(3.)空集在、集$E$与算集
不含任何元素的集合称为空集,记作$\varnothing$。空集是唯一存在的,且是任何集合的子集。在一个具体问题中,如果所涉及的集合都是某个集合的子集,则称这个集合为全集,记作$E$。设$A$为集合,$A$的所有子集构成的集合称为$A$的幂集,记作$P(A)$,即
令$|S|$表示集合$S$中的元素个数,那么若$|A|=n$,则$|P(A)|=2^{n}$。
{(4.集合的基本运算和算律}
集合的基本运算是并$(U)$、交$(\cap)$、相对补$(-)$、绝对补$(\sim)$和对称差$(\oplus)$,分别定义如下:
集合的基本运算遵从下述算律:
(1)槖等律
(2)结合律
(3)交换律
(4)分配律
(5)同一律
(6)零律
(7)排中律
(8)矛盾律
(9)吸收律
(10)德摩根律
(5.)传辚余的计数
解决有穷集合的计数问题有两种方法:文氏图和包含排斥原理。
设$S$为有穷集,$p_{1},p_{2},\cdots,p_{m}$是$m$条性质。$S$中的任何元素$x$对于性质$p_{i}(i=1$,$2,\cdots,m)$具有或者不具有,两种情况必居其一。令$\overline{A_{i}}$表示$S$中不具有性质$p_{i}$的元素构成的集合,那么包含排斥原理可表述为下面两个公式:
6.)小绱
通过本章的学习应该达到下面的基本要求。
能够正确地表示一个集合,会画文氏图。能判定元素是否属于给定的集合。
能判定两个集合之间是否存在包含、相等或真包含的关系。
能熟练进行集合的并$(U)$、交$(\cap)$、相对补$(-)$、绝对补$(\sim)$、对称差$(\oplus)$运算;会计算冥集$P(A)$。
求解与有穷集合计数相关的实际问题。
{第4章二元关系和函数}
{内容提要}
{(1.)有序对与笛卡儿积}
由两个元素$x$和$y$(允许$x=y$)按一定的顺序排列成的二元组称为一个有序对(也称序偶),记作$\langlex,y\rangle$。其中$x$是它的第一元素,$y$是它的第二元素。两个有序对$\langlex,y\rangle$与$\langleu,v\rangle$相等的充分必要条件是$x=u$且$y=v$。
设$A、B$为集合,$A$与$B$的笛卡儿积记作$A\timesB$,其中
笛卡儿积运算具有下述性质:
{2.关系、从$A$到$B$的关系利$A$上的关系}
如果一个集合为空集或者它的元索都是有序对,则称这个集合是一个二元关系,记作$R$。对于二元关系$R$,如果$\langlex,y\rangle\inR$,则记作$xRy$;如果$\langlex,y\rangle\notinR$,则记作$xRy$。
设$A、B$为集合,$A\timesB$的任何子集所定义的二元关系称作从$A$到$B$的二元关系,特别当$A=B$时,则称为$A$上的二元关系。当$A$含有$n$个元素,即$|A|=n$时,$A$上有$2^{2^{n}}$个不同的二元关系,其中最常用的$A$上的二元关系有下述5种。
{3.关系表示法}
表示关系的方法有3种:集合表达式、关系矩阵和关系图。其中,关系图只能表示有穷集$A$上的关系,关系矩阵可以表示有穷集$A$到$B$的关系与$A$上的关系。
{(4.关系的性质}
对于集合$A$上的关系$R$可以定义5种性质:自反性、反自反性、对称性、反对称性和传递性。
判别关系性质的方法如表4-1所示,其中的$\boldsymbol{M}^{2}$表示矩阵$\boldsymbol{M}$和$\boldsymbol{M}$相乘。注意在做乘法时的相加为逻辑加,即$0+0=0,0+1=1+0=1+1=1。\boldsymbol{M}-\boldsymbol{M}^{2}$表示将$\boldsymbol{M}$中的每个元素减去$\boldsymbol{M}^{2}$中的相对应元素后得到的结果矩阵,这里的减法是普通的减法。
充要条件&自反&反自反&对称&反对称&传䏲\\
{5.等价关系利划分}
设$R$为非空集合$A$上的关系,如果$R$是自反的、对称的和传递的,则称$R$为$A$上的等价关系。对任何$x,y\inA$,如果$<x,y>\in$等价关系$R$,则记作$x\simy$。对于$A$的任何元素$x,A$中与$x$等价的元素构成了$x$的等价类,记作$[x]_{R}$,简记作$[x]$,即
$A$上等价关系$R$的所有等价类的集合称为$A$在$R$下的商集,记作$A/R$,即
设$A$是非空集合,如果存在一个$A$的子集族$\pi(\pi\subseteqP(A))$,满足以下条件:
(2)$\pi$中任意两个元素不交;
(3)$\pi$中所有元素的并集等于$A$。
则称$\pi$为$A$的一个划分,且称$\pi$中元素为划分块。
可以证明$A$关于等价关系$R$的商集$A/R$就是$A$的划分;反之,反给定$A$的划分$\pi$,将$\pi$中划分块作为等价类也可以导出$A$上的等价关系。$A$上的等价关系与$A$的划分是一一对应的。6.倨席炎系与偏㙂集
设$R$为非空集合$A$上的关系,如果$R$是自反的、反对称的和传递的,则称$R$为$A$上的偏序关系,简称偏序,记作$\leqslant$。集合$A$和$A$上的偏序关系$(\leqslant)$一起称为偏序集,记作$<A,\leqslant>$。$\forallx,y\inA,x$与$y$之间只能保持下面4种关系之一:$x=y,x<y,y<x,x$与$y$不可比。这里的$x<y、y<x$以及$x$与$y$不可比的含义如下:
当$x<y$且不存在其他的元素$z$使得$x<z<y$成立时,称$y$盖住$x$。$x<y$意味着在偏序关系上$y$排在$x$的后边;而$y$盖住$x$则意味着在偏序关系上$y$紧跟在$x$的后边。
有穷集上的偏序可以用哈斯图来表示。在哈斯图中的元素是分层排列的。最底层是所有的极小元,相邻两层之间较高一层的元素至少盖住较低一层的一个元素。每条路径的最高层元素都是极大元。如果偏序集只有唯一的极小元,它就是该偏序集的最小元。类似地,如果偏序集只有唯一的极大元,它就是该偏序集的最大元。给定偏序集<$<A,\leqslant>$的子集$B$,如果存在元素$x\inA$大于或等于$B$中所有的元素,那么$x$就是$B$的上界。所有上界中的最小元就是$B$的最小上界。类似地,可以定义$B$的最大下界。$B$的最大下界或最小上界如果存在,一定是唯一的。
{(7.)获系运算}
和关系有关的运算有以下12种:
以下运算仅适合$A$上的关系$R$:
函数也称映射,它是一种特殊的二元关系。函数的定义:设$F$为二元关系,若对任意的$x\in\operatorname{dom}F$都存在唯一的$y\in\operatorname{ran}F$使得$xFy$成立,则称$F$为函数。若$<x,y>\in$函数$F$,则记作$y=F(x)$,称$y$是$F$在$x$的函数值。
给定集合$A、B$和函数$f$,若$f$满足下述条件:
{(9.)函数的性质}
某些函数$f:A\rightarrowB$具有单射、满射或双射的性质。这些性质分别定义如下:
(10.函数的复合利反函数
给定函数$f$和$g,f$与$g$的合成也是函数,称作$f$与$g$的复合函数,并且满足:
函数的逆不一定构成函数。但对于双射函数$f:A\rightarrowB$,它的逆$f^{-1}:B\rightarrowA$也是双射函数,称为$f$的反函数。
通过本章的学习应达到下面的基本要求。
能正确地使用集合表达式、关系矩阵和关系图表示给定的二元关系。
给定$A$上的关系$R$(可能是集合表达式,也可能是关系矩阵或关系图),能判别$R$的性质。
给定$A$上的等价关系$R$,求所有的等价类和商集$A/R$,或者求与$R$相对应的划分;给定$A$的划分$\pi$,求对应于$\pi$的等价关系$R$。
给定$A$上的偏序关系(),画出偏序集的哈斯图;给定偏序集$<A,\leqslant>$的哈斯图,求$A$和$\leqslant$的集合表达式。
确定偏序集的极大元、极小元、最大元、最小元、最大下界和最小上界。
给定集合$A、B$和$f$,判别$f$是否为从$A$到$B$的函数$f:A\rightarrowB$。如果是,说明$f:A\rightarrow$$B$是否为单射、满射、双射的。
应熟练掌握的计算:
给定集合$A$和$B$,求$A\timesB、B^{A}$,构造从$A$到$B$的双射函数。
在做以上计算时,如果没有特殊说明,所得结果应该与已知的关系或函数的表示方法一致。例如,已知关系$R$是用集合表达式给出的,那么,在计算$R^{-1}、R\uparrowA、R^{n}、r(R)、s(R)$、$t(R)$时所得的结果关系也要用集合表达式表示。若$R$用关系图给出,那么结果关系也应该用关系图给出。
{内容提要}
无向图与有向图无向图$G=<V,E>$,其中$V\neq\varnothing$称为顶点集,其元素称为顶点,$E$是$V\&V$的多重子集,称为边集,其元素称为无向边或边。有向图$D=<V,E>$,其中$V$同无向图,$E$是$V\timesV$的多重子集,其元素称为有向边或边。有时用$G$泛指图(无向的或有向的),但$D$只表示有向图。用$V(G)(V(D))、E(G)(E(D))$分别表示$G(D)$的顶点集与边集。
零图与平凡图只有顶点没有边的图称为零图,只有一个顶点的零图称为平凡图。
关联与相邻设图$G=\langleV,E>,u,v\inV,e=(u,v)\inE$(对于有向图,$e=\langleu,v>\epsilon$$E$),称$u、v$为$e$的端点(对于有向边,又称$u$为$e$的始点,$v$为$e$的终点),称$e$与$u、v$是彼此相关联的。无边关联的顶点称为孤立点。若$e$关联的两个顶点重合,则称$e$为环。若$u\neq$$v$,则称$e$与$u(v)$的关联次数为1。若$u=v$(即$e$为环),则称$e$与$u$关联的次数为2。若顶点$u、v$之间有边关联,则称$u$与$v$相邻。若两条边至少有一个公共端点(对于有向图,一条边的终点是另一条边的始点),则称这两条边相邻。
顶点的度数称无向图或有向图的顶点$v$作为边的端点的次数之和为$v$的度数或度,记作$d(v)$。称有向图的顶点$v$作为边的始点次数之和为$v$的出度,记作$d^{+}(v),v$作为边的终点的次数之和为$v$的人度,记作$d^{-}(v)$。显然,$d(v)=d^{+}(v)+d^{-}(v)$。称$\max\{d(v)\midv\inV(G)\}$为$G$的最大度,记作$\Delta(G)$或$\Delta$,称$\min\{d(v)\midv\inV(G)\}$为$G$的最小度,记作$\delta(G)$或$\delta$。类似地定义有向图的最大度$\Delta(D)$、最大出度$\Delta^{+}(D)$、最大人度$\Delta^{-}(D)$、最小度$\delta(D)$、最小出度$\delta^{+}(D)$、最小人度$\delta^{-}(D)$。
简单图对于无向图,若关联一对顶点的边多于一条,则称这些边为平行边。对于有向图,若关联一对顶点的方向相同的边多于一条,则称这些边为平行边。平行边的条数称作重数。既不含平行边,也不含环的图称为简单图。
完全图设$G$为$n$阶($n$个顶点)无向简单图,若$G$中任何两个顶点均相邻,则称$G$为$n$阶完全图,记作$K_{n}$。设$D$为$n$阶有向简单图,若$D$中任何两个顶点之间均有两条方向相反的边,则称$D$为$n$阶有向完全图。
正则图设$G$为$n$阶无向简单图,若$G$中每个顶点的度数均为$k$,则称$G$为$k$正则图。
{主要定理}
定理5.1(握手定理)任何图(无向图或有向图)中所有顶点的度数之和等于边数的2倍。任何有向图中所有顶点的人度之和等于所有顶点的出度之和等于边数。
推论任何图中奇度顶点的个数为偶数。
{2.通路、回路、图的连通性}
通路与回路设$\Gamma=v_{0}e_{1}v_{1}e_{2}\cdotse_{l}v_{l}$为图$G$中的顶点与边的交替序列,若$v_{i-1}、v_{i}$为$e_{i}$的端点(若$G$为有向图,要求$v_{i-1}$是$e_{i}$的始点,$v_{i}$是$e_{i}$的终点),$i=1,2,\cdots,l$,则称$\Gamma$为一条通路,$v_{0}、v_{l}$分别称为通路$\Gamma$的始点和终点,边的数目$l$称为$\Gamma$的长度。若通路的始点与终点重合,则称为回路。所有边互不相同的通路称为简单通路。所有边互不相同的回路称为简单回路。所有顶点互不相同的通路称为初级通路。所有顶点互不相同且所有边也互不相同的回路称为初级回路或圈。有边重复出现的通路称为复杂通路。有边重复出现的回路称为复杂回路。
顶点之间的连通关系在无向图$G$中,若顶点$u$到$v$有通路,则称$u$与$v$连通。规定顶点与自身连通。顶点之间的连通关系是等价关系。在有向图$D$中,若$u$到$v$有通路,则称$u$可达$v$。规定任何顶点与自身可达。
无向图的连通性若无向图$G$中任何两个顶点都连通,则称$G$是连通图。对于无向图$G$,设$V_{1},V_{2},\cdots,V_{k}$是顶点集$V$关于连通关系的等价类,则称它们的导出子图为$G$的连通分支,$G$的连通分支数记作$p(G)$。
有向图的连通性若略去有向图$D$中各边的方向所得无向图是连通图,则称$D$是弱连通图(或连通图);若$D$中任何两个顶点至少一个可达另一个,则称$D$是单向连通图;若$D$中任何两个顶点都是相互可达的,则称$D$是强连通图。强连通图一定是单向连通图,单向连通图一定是弱连通图。
{3.图的短䧏表示}
无向图的可达矩阵和邻接矩阵与有向图的可达矩阵和邻接矩阵类似,实际上,只要把每条无向边看作一对方向相反的有向边,就可以把无向图作为有向图的特殊情况。无向图的可达矩阵和邻接矩阵都是对称的。
{主要定理及推论}
{4.最短路径问题}
最短路径设带权图$G,u、v$为$G$中两个顶点,从$u$到$v$所有通路中权最小的通路称为$u$到$v$的最短路径,其权称作$u$到$v$的距离。
最短路径问题是求带权图中指定两点之间的最短路径及距离。Dijkstra标号法是最短路径问题的常用有效算法,它适用于所有的权非负的情况。
{5.项目网络图与关旔路径}
项目网络图是一个带权的有向图,用来描述项目中活动的完成时间及相互关系。项目网络图中从始点到终点的最长路径称作关键路径。关键路径上的活动称作关键活动。通过计算各顶点的最早开始时间和最晩完成时间找到关键路径及活动的相关数据。
{6.着色问题}
着色给无环的无向图的每个顶点涂一种颜色,使得相邻的顶点涂不同的颜色,称作图的点着色,简称着色。图的着色问题是如何用尽可能少的颜色给图着色。
本章概念较多,它们是图论中的基本概念。在学习和领会这些概念时,以下6点要特别注意。
(1)牢记握手定理及其推论,并且能灵活应用。例如,在求解无向图(例如,已知边数$m$和一些顶点的度数,求另外一些顶点的度数),求解无向树(见第7章)以及判断某些非负整数序列能否充当图的度数序列等问题中都要用到握手定理或推论。在图论的许多证明题中也要用到握手定理。
(2)记住简单图的概念和性质,如$n$阶无向简单图$G$的最大度$\Delta(G)\leqslantn-1,n$阶有向简单图$D$的最大度$\Delta(D)\leqslant2(n-1)$,最大出度$\Delta^{+}(D)\leqslantn-1$,最大人度$\Delta^{-}(D)\leqslantn-1$。在讨论给定的非负整数列能否充当无向图的度数序列时,都要用到以上性质。另外还要掌握完全图、正则图、补图等概念。
(3)清楚图同构的概念。对一些比较简单的情况,会根据定义和必要条件判断两个图是否同构。会画出4阶无向完全图$K_{4}$和3阶有向完全图的所有非同构的子图。
(4)清楚通路与回路的概念及其分类。初级通路(回路)都是简单通路(回路),但反之不真。长为1的圈是环,长为2的圈是两条平行边,只能在非简单图中出现。在简单图中初级回路(圈)的长度都大于或等于3。
(5)在讨论图的连通性时,要特别注意有向连通图的分类及它们之间的关系,即强连通的有向图必为单向连通的,单向连通的必为弱连通的,但反之都不真。
(6)在图的矩阵表示中,可以用邻接矩阵及各次幂,求图中的通路数及回路数。要注意,这里不同的通路(回路)是按定义来区分的,而不是同构意义下区分的。例如,长度为$l(l\geqslant1)$的有向圈在计算长度为$l$的回路时被计算$l$次,也就是说,不同始点(也是终点)的圈被看成是不同的。
{第6章特殊的图}
{内容提要}
{(1.)一部图}
若能将无向图$G=<V,E>$的顶点集$V$划分成两个不相交的非空子集$V_{1}$和$V_{2}$,使得$G$中任何一条边的两个端点都是一个属于$V_{1}$,另一个属于$V_{2}$,则称$G$为二部图,称$V_{1}$和$V_{2}$为互补顶点子集,记为$G=<V_{1},V_{2},E>$。若简单二部图$G=<V_{1},V_{2},E>$中$V_{1}$的每个顶点与$V_{2}$的每个顶点都相邻,则称$G$为完全二部图,记作$K_{r,s}$,其中$\left|V_{1}\right|=r,\left|V_{2}\right|=s$。
匹配与匹配数设$G=<V,E>$为无向图,$E^{\prime}\subseteqE$,若$E^{\prime}$中任何两条边均不相邻,则称$E^{\prime}$为$G$中的匹配。若$E^{\prime}$中再加人任何一条边都不再是$G$中的匹配,则称$E^{\prime}$为$G$中极大匹配。边数最多的匹配称为最大匹配。最大匹配中边的条数称为$G$的匹配数,记作$\beta_{1}(G)$,简记为$\beta_{1}$。设$M$为$G=<V,E>$中的匹配,$v\inV$,若$v$与$M$中的边关联,则称$v$为$M$饱和点,否则称$v$为$M$非饱和点。若$G$中所有的顶点都是$M$饱和点,则称$M$为$G$中的完美匹配。
{主要定理}
定理6.1无向图$G$为二部图当且仅当$G$中无奇数长度的回路。
Hall定理中的条件称为相异性条件。
定理6.3在二部图$G=<V_{1},V_{2},E>$中,若存在正整数$t$使得:
(1)$V_{1}$中每个顶点至少关联$t$条边;
(2)$V_{2}$中每个顶点至多关联$t$条边,
则$G$中存在$V_{1}$到$V_{2}$的完备匹配。
定理6.3中的条件称为$t$条件。
{2.欧接图}
欧拉回路(通路)经过图中每条边一次且仅一次并且行遍图中所有顶点的回路(通路),称为欧拉回路(通路)。有欧拉回路的图称为欧拉图。
{主要定理}
定理6.4无向图$G$有欧拉回路当且仅当$G$连通且无奇度顶点。
定理6.5无向图$G$有欧拉通路,但无欧拉回路,当且仅当$G$连通且恰好有两个奇度顶点。这两个奇度顶点是每条欧拉通路的两个端点。
定理6.6有向图$D$有欧拉回路当且仅当$D$连通且每个顶点的人度等于出度。
定理6.7有向图$D$有欧拉通路,但无欧拉回路,当且仅当$D$连通,且除两个顶点外,其余顶点的人度等于出度,这两个例外的顶点中,一个的人度比出度大1,另一个的人度比出度小1。
{3.哈管顸泈}
哈密顿回路与哈密顿通路经过图中每个顶点一次且仅一次的回路(通路)称为哈密顿回路(通路)。有哈密顿回路的图称为哈密顿图。
{主要定理}
若$G$中有哈密顿通路,则
定理6.9设$G$为$n(n\geqslant3)$阶无向简单图,若$G$中任何一对不相邻的顶点度数之和都大于或等于$n-1$,则$G$中有哈密顿通路;若$G$中任何一对不相邻的顶点度数之和都大于或等于$n$,则$G$中有哈密顿回路。
平面图与平面嵌入如果能将无向图$G$画在平面上,使其除在顶点处外没有边相交,则称$G$为平面图。画出的无边相交的图称为$G$的平面嵌人。
平面图的面与次数平面图$G$的平面嵌人中的边将平面分成若干区域,每个区域称为$G$的一个面,其中有一个面积无限的面称为无限面或外部面,其余面积有限的面称为有限面或内部面。包围一个面的所有边构成的回路称为该面的边界,边界的长度称为面的次数,面$R$的次数记作$\operatorname{deg}(R)$。这里所谈回路可能是初级的,也可能是简单的、复杂的,还可能是几条回路。
极大平面图如果在简单平面图$G$的任意两个不相邻的顶点之间再加一条边,所得图为非平面图,则称$G$为极大平面图。
极小非平面图若在非平面图$G$中任意删除一条边,所得图为平面图,则称$G$为极小非平面图。
平面图的对偶图设$G$是一个平面图的平面嵌人,构造图$G^{*}$如下:在$G$的每个面$R_{i}$中放置一个顶点$v_{i}^{*}$。对$G$的每条边$e$,若$e$在$G$的面$R_{i}$与$R_{j}(i\neqj)$的公共边界上,则作边$e^{*}=\left(v_{i}^{*},v_{j}^{*}\right)$与$e$相交,且不与其他任何边相交。若$e$为$G$中的桥且在面$R_{i}$的边界上,则作以$v_{i}^{*}$为端点的环$e^{*}=\left(v_{i}^{*},v_{i}^{*}\right)$与$e$相交,且不与其他任何边相交,称$G^{*}$为$G$的对偶图。
地图着色地图是连通的无桥平面图的平面嵌人,每个面是一个国家。对地图的每个国家涂一种颜色,使相邻的国家涂不同的颜色,称为地图着色。地图着色问题就是要用尽可能少的颜色给地图着色。地图着色可以转化成平面图的点着色。
{主要定理}
定理6.10平面图的所有面的次数之和等于边数的2倍。
定理6.11极大平面图是连通的。
定理6.12设$G$是$n(n\geqslant3)$阶简单的连通平面图,则$G$为极大平面图当且仅当$G$的每个面的次数均为3。
定理6.13(欧拉公式)设$G$为连通的平面图,则有
其中,$n、m、r$分别为$G$的顶点数、边数和面数。
定理6.14(欧拉公式的推广)设平面图$G$有$p$个连通分支,则有
定理6.15设$n$阶连通平面图$G$有$m$条边,每个面的次数至少为$l(l\geqslant3)$,则
若$G$有$p$个连通分支,其他条件不变,则
定理6.16(库拉图斯基定理)图$G$为平面图当且仅当$G$中没有可以收缩成$K_{5}$或$K_{3,3}$的子图。子图。
定理6.17(库拉图斯基定理)图$G$为平面图当且仅当$G$中没有与$K_{5}$或$K_{3,3}$同胚的
定理6.18任何平面图都是4-可着色的。
本章介绍4种特殊的图,在学习这些特殊的图时应注意以下3点。
(1)弄清完美匹配与完备匹配的区别。
(2)注意定理6.8是有哈密顿回路或哈密顿通路的必要条件,而不是充分条件。例如,彼德森图满足定理中条件,但它不是哈密顿图。而定理6.9中的条件是有哈密顿回路或哈密顿通路的充分条件,但不是必要条件。例如,$n(n\geqslant5)$阶圈不满足这个条件,但$n$阶圈为哈密顿图。
(3)注意$K_{5}$和$K_{3,3}$在平面图理论中的特殊地位,掌握库拉图斯基定理。
{第7章树}
{内容提要}
{(1.)无向树及生成树}
无向树连通不含回路(初级回路或简单回路)的无向图称为无向树,常用$T$表示。每个连通分支都是无向树的非连通无向图称为森林。在树$T$中,度数为1的顶点称为树叶,非树叶的顶点称为分支点。平凡图称为平凡树,它没有树叶,也没有分支点。
生成树若无向图$G$的生成子图$T$是一棵树,则称$T$为$G$的生成树。$G$在$T$中的边称为$T$的树枝,$G$不在$T$中的边称为$T$的弦。$T$的全体弦组成的集合的导出子图称为$T$的余树。注意,$T$的余树不一定是树,它可能不连通,也可能含回路。
基本回路与基本回路系统设$T$是无向图$G$的生成树,对每条弦$e,G$中有唯一一条由$e$和$T$的树枝构成的初级回路,称为对应弦$e$的基本回路。$G$中所有基本回路的集合称为对应$T$的基本回路系统。
基本割集与基本割集系统对$T$的每条树枝$a,G$中有唯一一个由$a$和$T$的弦构成的割集,称为对应树枝$a$的基本割集。$G$中所有基本割集的集合称为对应$T$的基本割集系统。
最小生成树无向带权连通图$G$的权最小的生成树称为最小生成树。可用避圈法(Kruskal算法)求最小生成树。
{主要定理}
定理7.1设无向图$G=<V,E>,|V|=n,|E|=m$,则下面命题等价。
(1)$G$连通且不含回路,即$G$是一棵树。
(2)$G$的每对顶点之间有唯一的一条路径。
(5)$G$中无回路,但在$G$中任何两个不相邻顶点之间加一条新边,所得图中含唯一的一条初级回路。
(6)$G$连通且每条边都是桥。
定理$7.2n(n\geqslant2)$阶无向树至少有两片树叶。
定理7.3任何连通的无向图$G$都有生成树。
定理7.4设$T$是$n$阶$m$条边无向连通图$G$中的生成树,则$T$有$n-1$条树枝、$m-n+1$条弦。
{2.根树及其应用}
有向树及根树若略去有向图所有边的方向所得无向图为无向树,则称$D$为有向树。一棵非平凡的有向树$T$,如果有一个顶点的人度为0,其余顶点的人度均为1,则称$T$为根树。在根树中,人度为0的顶点称为树根。人度为1、出度为0的顶点称为树叶。人度为1、出度不为0的顶点称为内点。树根与内点统称为分支点。在根树中,从树根到一个顶点的通路长度称为该顶点的层数。顶点的最大层数称为树高。
家族树一棵根树可被看成一个家族。若在树中有有向边$\langleu,v>$,则称$u$是$v$的父亲,$v$是$u$的儿子。若$v_{1}、v_{2}$的父亲相同,则称它们是兄弟。又若$u$可达$v$,则称$u$为$v$的祖先,$v$为$u$的后代。
根子树设$v$为根树$T$中非根顶点,称由$v$及其后代的导出子图为$T$的以$v$为根的根子树。
有序树若对根树$T$中每层上的顶点指定顺序,则称$T$为有序树。
根树的分类设$T$为一棵根树。
(1)若$T$的每个分支点至多有$r$个儿子,则称$T$为$r$叉树。若$T$的每个分支点都恰好有$r$个儿子,则称$T$为$r$叉正则树。此时又若$T$的所有树叶层数相同,则称$T$为$r$叉完全正则树。
(2)有序的$r$叉树,称为$r$叉有序树。有序的$r$叉正则树称为$r$叉有序正则树。有序的$r$叉完全正则树称为$r$叉有序完全正则树。
一棵带$t$片树叶的二叉树可以产生一个含$t$个符号串的二元前缀码。给定字符串出现的频率,使得编码期望长度最小的前缀码称作最佳前缀码。可以用以频率为权的最优二叉树产生最佳前缀码。
{3.二叉有序树与算式}
可以用二叉有序树表示算式,做法如下:运算符放在分支点上,数或变量放在树叶上,每个运算符的运算对象放在它的子树上,并规定被减数和被除数放在左子树上。
行遍(周游)二叉有序树
(1)中序行遍法访问次序为左子树、树根、右子树。
(2)前序行遍法访问次序为树根、左子树、右子树。
(3)后序行遍法访问次序为左子树、右子树、树根。
其中,左子树或右子树可以缺省。对表示算式的二叉有序树采用中序行遍法可以还原算式。用前序行遍法可以产生波兰符号法。用后序行遍法可以产生逆波兰符号法。(4.)小经算
学习本章要注意以下4点。
(1)在求解无向树时,一定注意将树的主要性质之一的$m=n-1(m、n$分别为树的边数和顶点数)与握手定理配合在一起用,即
此公式在解无向树时起很大作用。
(2)画$n$阶非同构的无向树时,也要用到树的性质$m=n-1$。由此就知道了所求树的度数之和,因而能给出不同度数序列的分配方案。在写度数序列时注意非平凡树所有顶点的度数都大于或等于1且小于或等于$n-1$。根据不同的度数序列画出的无向树是非同构的。但同一个度数序列,由于顶点之间的相邻关系的不同,可能产生多个非同构的树。
(3)画$n$阶非同构的根树时,要先画出$n$阶非同构的无向树,然后由每个无向树再派生出非同构的根树,就可以得到全体$n$阶非同构的根树了。
(4)在用Huffman算法求最佳前缀码时,若先将各符号出现频率乘100,所得数作为权求最优树,则最优树的权$W(T)$为传输100个按给定频率出现的符号所用二进制数字的期望个数。另外,还应注意,最优树不一定唯一,因而所得前缀码可能不同。
{第8章组合分析初步}
{内容提要}
{(1.)加法法则利乘法法则}
加法法则如果事件$A$有$p$种产生的方式,事件$B$有$q$种产生的方式,则事件“$A$或$B$”有$p+q$种产生的方式。
乘法法则如果事件$A$有$p$种产生的方式,事件$B$有$q$种产生的方式,则事件“$A$与$B$”有$pq$种产生的方式。
加法法则与乘法法则可以推广到$n$个事件。
{2.排列与组合的定义}
设$S$为$n$元集。从$S$中有序选取的$r$个元素称为$S$的一个$r$排列,不同排列的总数记作$Pn$。如果$r=n$,则称这个排列为$S$的全排列,简称$S$的排列。
设$S$为$n$元集,从$S$中无序选取的$r$个元素称为$S$的一个$r$组合,不同组合的总数记作$C_{n}^{r}$。
的$a_{i}$以供选取。
从多重集$S$中有序选取的$r$个元素称为$S$的一个$r$排列。当$r=n_{1}+n_{2}+\cdots+n_{k}$时,称为$S$的全排列,也称$S$的排列。
从多重集$S$中无序选取的$r$个元素,也就是$S$的一个$r$个元素的子多重集,称为$S$的一个$r$组合。
(3.排列组合的基本公式
(1)集合的排列组合公式。
{(4.)邀推方程解求解力法}
通过递推方程和初值求得函数$a_{n}$的显示表达式(非递归表示)称为求解递推方程。
递推方程的求解方法有迭代归纳法和递归树法。这些方法的基本思想:对于所有的$n,n-1,n-2\cdots$不断用方程的右部替换表达式或者递归树中的函数项,直到初值为止,从而得到一系列的项之和。然后通过求和公式把这个和求出来,或者估计出这个和的渐近的上界。所得到的解是否正确,可以通过归纳法进行验证。
递推方程在递归算法的分析中有着重要的应用。分治算法的递推方程通常具有下述形式:设$a、b$为正整数,$n$为问题的输人规模(不妨设$n=b^{k}$),$n/b$为子问题的输人规模,$a$为子问题的个数,$d(n)$为将原问题分解成子问题以及将子问题的解综合得到原问题解的代价。那么算法的时间复杂度函数满足下述递推方程:
当$d(n)=c$时,$c$代表某个常数,则该递推方程的解是
当$d(n)=cn$时,$c$代表某个常数,则该递推方程的解是
通过本章的学习应该达到下面的基本要求:使用加法法则、乘法法则等计数规则进行组合计数。正确使用排列、组合、多重集排列、多重集组合公式解决实际的计数问题。能够针对实际计数问题确定相应的递推方程和初值,并加以求解。
{第9章代数系统简介}
{内容提要}
{(1.)二元利一元代数运算}
设$S$为集合,函数$f:S\timesS\rightarrowS$和$f:S\rightarrowS$分别称为$S$上的二元和一元运算。若$f$是$S$上的二元或一元运算,这时也称$S$对运算$f$是封闭的。通常用不同的算符,如。,*,-$\Delta,\cdots$来代表不同的二元或一元运算。
一个二元或一元运算的方法有两种一一解析表达式或运算表,其中运算表只能定义有穷集上的二元或一元运算。
二元运算的性质
设。和*为$S$上的二元运算,和这些运算相关的性质(或称算律)如下。
视审徒
上述的交换律、结合律、幂等律和消去律都是对•运算而言的,其中消去律中的$\theta$指该运算的零元。剩下的两条算律是与。和*两个运算有关的。注意在谈分配律时应该说明哪个运算对哪个运算可分配,因为当・运算对*运算满足分配律时,*运算对。运算却不一定满足分配律。
{3.二元运算的特异元素}
设॰为$S$上的二元运算,和•运算相关的特异元素有么元$\boldsymbol{e}$
幂等元$x$
对于给定的集合$S$和$S$上的二元运算॰,如果存在么元或零元,一定是唯一的;如果存在幂等元和可逆元,则可能存在多个。对于可结合的二元运算,如果$S$中的某个元素$x$是可逆元,则$x$存在唯一的逆元,记作$x^{-1}$。特别地,么元$e$是可逆元且$e^{-1}=e$,而零元$\theta$不是可逆元。
{4.代数系统、子代数利积代数}
在某些代数系统中将一些二元运算的特异元素作为系统性质规定下来,例如,独异点中的么元、布尔代数中的全下界0和全上界1等,称这些元素为该系统的代数常数。
(5.代数系统的同态与闹构
应群、独㫒点和群的一般概念
设$V=<S,\circ>$是代数系统,$\circ$为二元运算。如果$○$运算是可结合的,则称$V$为半群。如果半群中的॰运算含有么元$e$,则称该半群为含么半群,也称独异点。为了强调么元的存在,有时将独异点$V$记作$<S,\cdot,e>$。设$<G,\circ>$是独异点,如果对$G$中的任何元素$x$都有$x^{-1}\inG$,则称$G$是群。由以上定义可知,群一定是独异点和半群,但半群和独异点不一定是群。
在半群中可以定义元素的正整数饮幂。对任意元素$x$和正整数$n$有
表示$n$个$x$运算的结果。除此之外,在独异点和群中可以定义$x$的零饮幕,即$x^{0}=e_{0}$。进一步,在群中还可以定义$x$的负整数次幕。设$n$为正整数,那么
表示$n$个$x^{-1}$运算的结果。半群、独异点和群的幂运算都遵从下面的规则:
{(7.)保计管用木结利典型实例}
若群$G$中的二元运算是可交换的,则称群$G$为交换群,也称阿贝尔(Abel)群。
若群$G$中有无限多个元素,则称$G$为无限群,否则称为有限群。对有限群$G,G$中元素的个数叫作$G$的阶,记作$|G|$。
只含么元$e$的群称为平凡群,是1阶群。
下面是一些典型群的实例。
(1)整数集$\mathbf{Z}$、有理数集$\mathbf{Q}$、实数集$\mathbf{R}$和复数集$\mathbf{C}$关于数的加法构成群,分别称为整数加群、有理数加群、实数加群和复数加群。非零实数集$\mathbf{R}^{*}$关于数的乘法构成群。这些群都是无限群,也是阿贝尔群。
(3)设$G=\{e,a,b,c\},G$上的二元运算由表9-1给出。不难证明$G$是一个群,称为Klein四元群。从表中可以看出$G$中运算是可交换的,$e$为么元,$x\inG,x^{-1}=$$x$,且在$a、b、c$这3个元素中任何两个元素的运算结果都等于剩下的元素。
(4)设$G$为群,如果存在$a\inG$使得
则称$G$为循环群,记作$G=<a>$,称$a$为$G$的生成元。若循环群$G$中含有无限多个元素,则称$G$为无限循环群;若$|G|=n$,则称$G$为$n$阶循环群。容易证明循环群都是阿贝尔群,但阿贝尔群不一定是循环群。例如,Klein四元群是阿贝尔群,但不是循环群。
(5)设$S=\{1,2,\cdots,n\}。S$上的任何双射函数$\sigma:S\rightarrowS$称为一个$n$元置换,置换的复合运算称为置换的乘法。若将$S$上所有$n$元置换的集合记作$S_{n}$,那么$S_{n}$关于置换的乘法构成群,称为$n$元对称群。$S_{n}$的任何子群称为$n$元置换群。当$n\geqslant3$时,$S_{n}$不是阿贝尔群。对任何$n$元置换$\sigma\inS_{n}$,可以将$\sigma$记为
称为$\sigma$的置换表示。若$n$元置换$\tau$的映射规则满足
并且保持其他的元素不变,可将$\tau$简记为
称为一个$m$阶轮换。可以证明任何$n$元置换$\sigma$都可以唯一地表示成一系列不相交的轮换之积,称为$\sigma$的轮换表示。
{8.元素的阶}
设$G$为群。$x\inG$,使得等式$x^{k}=e$成立的最小正整数$k$称为$x$的阶。如果$x$的阶存在,记作$|x|$,并称$x$是有限阶元,否则称$x$为无限阶元。
设$G$是无限群,那么$G$中可能存在着无限阶元。例如,整数加法群$<\mathbf{z},+>$,除0以外,其他元素都是无限阶元。但对某些无限阶群来说,尽管群中含有无限多个元素,但每个元素都是有限阶元。例如,单位根构成的集合
关于数的乘法构成群。对任意$x\inG$,若$x$是$n$次根,则$|x|=n$。
若$G$是$n$阶群,则$G$中每个元素的阶都存在,并且是$n$的因子。
{(9.般的基本性质}
关于群的性质有以下定理。
定理6.1设$G$为群,$n,m$为整数,则群中的幂运算满足:
定理6.3设$G$为群,则$G$中适合消去律,即对任意$a,b,c\inG$有
设$G$是群,$H$是$G$的非空子集,如果$H$关于$G$中的运算构成群,则称$H$为$G$的子群,记作$H\leqslantG$。任何群$G$都有两个平凡子群:$\{e\}$和$G$自己,除此之外都是$G$的非平凡的真子群。
设$G$为群,$x\inG$,称$x$的所有幂的集合
所构成的子群为由$x$生成的子群,记作$\langlex\rangle$。
设$G$为群,令
即与$G$中所有元素都可交换的元素构成的集合,则$C$是$G$的子群,称为$G$的中心。
设$<R,+,\cdots>$是代数系统,十和-为二元运算,分别称为加法和乘法。若
(1)$<R$,十$>$为阿贝尔群;
(3)乘法(-)对加法$(+)$适合分配律。
由于在环$R$中存在两个二元运算,为了避免混淆,通常将加法么元记作0,而将乘法么元记作1(如果存在)。类似地,可将环中元素$a$的加法逆元称为$a$的负元,记作一$a$;而将$a$的乘法逆元称为$a$的逆元,记作$a^{-1}$。
乘法可交换的,含有乡元1的,并且没有左零因子和右零因子的环称为整环。
如果整环$R$至少含有两个元素,且每个元素$x(x\neq0)$都有逆元$x^{-1}\inR$,则称$R$是域。
有理数集$\mathbf{Q}$、实数集$\mathbf{R}$、复数集$\mathbf{C}$关于数的加法和乘法分别构成有理数域、实数域和复数域。但整数集$\mathbf{Z}$关于数的加法和乘法只能构成整环,但不是域。模$n$整数环$<\mathbf{Z}_{n},\oplus$,$\odot>$当$n$为合数时不是整环,也不是域;但当$n$为索数时构成域。
(12)格的肉个等价定义
设$<S,\leqslant>$是偏序集,若$\forallx,y\inS,\{x,y\}$都有最小上界和最大下界,则称$S$关于$\leqslant$构成一个格。由于最小上界与最大下界的唯一性,可以把求$\{x,y\}$的最小上界和最大下界看成$x$与$y$的二元运算,分别用算復$\vee$和$\wedge$表示,从而$<S,\vee,\wedge>$构成一个具有两个二元运算的代数系统,称为由偏序集的格所导出的代数系统。
设$<S$,*$0>$是具有两个二元运算的代数系统,且对于*和。运算适合交换律、结合律和吸收律,则可以适当定义$S$中的偏序$(\leqslant)$使得$<S,\leqslant>$构成一个格,且$\foralla,b\inS$,有
称这个格是由代数系统$<S$,,,$\circ>$导出的格。
以上两种定义格的方法是等价的。
{(3.)格的性质}
格的主要性质有以下两条。
(1)格的对偶原理。设$f$是含有格中元素以及符号$=、\leqslant、\geqslant、V、\wedge$的命题。令$f^{*}$是将$f$中的$\leqslant$改写成$\geqslant、\geqslant$改写成$\leqslant、V$改写成$\Lambda、\Lambda$改写成$V$所得到的命题,称为$f$的对偶命题。根据格的对偶原理,若$f$对一切格为真,则$f^{*}$也对一切格为真。
(2)设$<L,\leqslant>$为格,则运算$\vee$和$\wedge$适合交换律、结合律、幂等律和吸收律。
(4.)分配格、你补格利布尔格
成立,则称$L$为分配格。
如果格$L$中存在最小元和最大元,则分别称为$L$的全下界和全上界,记作0和1。这时也称$L$为有界格,记作$\langleL,\Lambda,V,0,1>$。
设$L$为有界格,$x\inL$,若存在$y\inL$使得$x\wedgey=0$且$x\veey=1$成立,则称$y$是$x$的补元。在有界格中,0和1互为补元,而其他元素则情况各异,有的不存在补元,有的存在一个补元,有的存在多个补元。如果有界格中的每个元素都至少存在一个补元,则称这个格为有补格。
有补分配格称为布尔格,也称布尔代数。在布尔代数$B$中每个元素都存在唯一的补元,求补运算'可看成布尔代数中的一元运算,并满足下述算律:
通过本章的学习应该达到下面的基本要求。
给定集合与运算的解析表达式,写出该运算的运算表。
给定集合和运算,判别该集合对运算是否封闭(或者说运算是否为给定集合上的运算,也可以说给定集合对于这些运算是否构成代数系统)。
给定二元运算,说明运算是否满足交换律、结合律、幂等律、分配律和吸收律。
给定二元运算,求出该运算的么元、零元、幂等元和所有可逆元素的逆元。
给定集合$S$和二元运算॰,能判定$<S,\circ>$是否构成半群、独异点和群。
给定半群$S$(或独异点$V$)和子集$B$,判定$B$是否为$S$的子半群($V$的子独异点);给定群$G$和子集$H$,判定$H$是否为$G$的子群。
给定群$G$和$x\inG$,求$|G|$、$|x|$以及$x^{n}$。求解群方程。求由$x$生成的子群$<x>$。求循环群$G=<a>$的所有生成元和子群。
给定$n$元置换$\sigma$和$\tau$,试把它们表成不交的轮换之积,求$\sigma\tau$和$\sigma^{-1}$。
给定集合$S$和$S$上的两个二元运算,判定它们能否构成环、交换环、含么环、整环和域。计算环中的多项式。判别格、分配格、有界格、有补格和布尔格。
求格中公式的对偶式。给定格中元素$x、y$,求$x\wedgey$和$x\veey$。求有界格的全下界、全上界和给定元素的补元。
{第10章形式语言和自动机初步}
{内容提要}
{(1.形式语享䅐形式文法}
字母表与字符串字母表是一个非空的有穷集合。由字母表$\Sigma$中的符号组成的有穷序列称为字母表$\Sigma$上的字符串。字符串$\omega$中的符号数称为$\omega$的长度,记作$|\omega|$。长度为0的字符串称为空串,记作$\varepsilon。n$个$a$组成的字符串$aa\cdotsa$记作$a^{n}$。
子串、前缀与后缀字符串$\omega$中若干连续的符号组成的字符串称为$\omega$的子串。从最左端开始的子串称为前缀。在最右端结束的子串称为后缀。
语言字母表$\Sigma$上的字符串全体记作$\Sigma^{*}$。$\Sigma^{*}$的任何子集称为字母表$\Sigma$上的形式语言,简称语言。
文法形式文法简称文法,它由4部分组成,记作$G=\langleV,T,S,P>$,其中$V$是有穷的变元集,变元又称为非终极符;$T$是有穷的终极符集,$T\capV=\varnothing;S\inV$称为起始符;$P$是有穷的产生式集,每个产生式形如$\alpha\rightarrow\beta$,这里$\alpha,\beta\in(V\cupT)^{*}$且$\alpha\neq\varepsilon$。
文法生成的语言文法$G=<V,T,S,P>$生成的语言
著名的语言学家乔姆斯基(N.Chomsky)把文法分成4类,分别生成4个层次的语言,称为乔姆斯基谱系。分类如下所述。
0型文法与0型语言0型文法就是文法,又称为短语结构文法或无限制文法。0型文法生成的语言称为0型语言。
1型文法(上下文有关文法,CSG)与1型语言(上下文有关语言,CSL)如果文法的每个产生式$\alpha\rightarrow\beta$有$|\alpha|\leqslant|\beta|$,则称为1型文法,或上下文有关文法。如果存在1型文法$G$使得$L=L(G)$或$L=L(G)\cup\{\varepsilon\}$,则称$L$是1型语言,或上下文有关语言。
2型文法(上下文无关文法,CFG)与2型语言(上下文无关语言,CFL)如果文法中每个产生式都形如$A\rightarrow\alpha$,其中$A\inV$,则称为2型文法,或上下文无关文法。2型文法生成的语言称为2型语言,或上下文无关语言。
3型文法(正则文法)与3型语言(正则语言)右线性文法与左线性文法统称为3型文法或正则文法。3型文法生成的语言称为3型语言或正则语言。
语法分析树又称派生树,用来描述CFG派生的有序树,它可以给出派生出的字符串的语义。
{2.有穷自动机}
确定型有穷自动机(DFA)及其接受的语言确定型有穷自动机简记作DFA,由5部分组成,记作$M=<Q,\Sigma,\delta,q_{0},F>$,其中$Q$是有穷的状态集,$\Sigma$是有穷的输人字母表,$\delta:Q\times\Sigma\rightarrowQ$是状态转移函数,$q_{0}\inQ$是初始状态,$F\subseteqQ$是接受状态集或终结状态集。
非确定型有穷自动机(NFA)非确定型有穷自动机$M=<Q,\Sigma,\delta,q_{0},F>$与确定型有穷自动机的区别是状态转移函数为$\delta:Q\times\Sigma\rightarrowP(Q)$,这里$P(Q)$是$Q$的幂集。
如果把状态$q$等同于单元集$\{q\}$,则DFA是NFA的特殊情况。DFA和NFA统称为有穷自动机,简记作FA。
带$\varepsilon$转移的NFA对NFA稍加推广,不仅在读$\Sigma$的符号后做状态转移,而且可以在不读任何符号(或说读空串$\varepsilon$)的情况下自动做状态转移,即状态转移函数为$\delta:Q\times(\Sigma\cup\{\varepsilon\})$$\rightarrowP(Q)$,这就是带$\varepsilon$转移的$NFA$。
状态转移图DFA可以用状态转移图表示。状态转移图是一个有向图,每个结点代表一个状态。初始状态用一个指向该结点的箭头标明,接受状态用双圈标明。如果$\delta(q,a)=q^{\prime}$,则从结点$q$到$q^{\prime}$有一条弧,并且在弧旁标明$a$。NFA的状态转移图与DFA的类似,两者的区别如下:对于每个$q\inQ$和$a\in\Sigma,\mathrm{DFA}$的状态转移图中恰好有一条从结点$q$出发标有符号$a$的弧,而NFA的状态转移图中可以有一条或多条这样的弧,也可以没有这样的弧。
{(3.)此则表达式}
连接设$L_{1},L_{2}$是字母表$\Sigma$上的语言,记
闭包设$L$是字母表$\Sigma$上的语言,记
$L^{*}$称为$L$的闭包,$L^{+}$称为$L$的正闭包。
正则表达式及其表示的语言
(3)每个$a\in\Sigma$是正则表达式,它表示$\{a\}$;
(4)如果$r$和$s$分别是表示语言$R$和$S$的正则表达式,则$(r+s)、(r\cdots)$和$(r*)$也是正则表达式,它们分别表示$R\cupS、R\cdotS$和$R$*;
(5)有限次运用上述规则得到的表达式是正则表达式。
正则表达式$\alpha$表示的语言记作$\langle\alpha>$。
规定运算的优先等级:*,,,
设想TM是由控制器、读写头及一条带组成的装置。带的两头是无穷的,被划分成无穷多个小方格,每个小方格内存放$\Gamma$中的一个符号。控制器处于$Q$中某个状态。读写头扫视一个方格,可以读取和改写这个方格的内容,向左或向右移动。假设$M$的当前状态是$q$,读写头读到的符号是$s$。如果$\delta(q,s)=\left(s^{\prime},L,q^{\prime}\right)$,则读写头把扫视的方格内的符号改写成$s^{\prime}$,向左移动一格,控制器转移到状态$q^{\prime}$;如果$\delta(q,s)=\left(s^{\prime},R,q^{\prime}\right),M$的动作与刚才一样,只是读写头向右移动一格;如果$\delta(q,s)$没有定义,则停机。
格局带上的内容,读写头扫视的位置和控制器的状态称为TM$M$的一个格局。TM的格局可写成$\alphaq\beta$,其中,$q\inQ,\alpha,\beta\in\Gamma^{*}$且$\beta\neq\varepsilon$。它表示带的内容为$\alpha\beta$,两头的其余部分均为$B$,控制器处于状态$q$,读写头扫视$\beta$左端的第一个復号。设当前的状态为$q$,读到的符号为$a$。如果$\delta(q,a)$没有定义,则称这个格局是停机格局。当$M$进人停机格局后,$M$停机,计算结束。如果$q\inA$且为停机格局,则称这是接受的停机格局。
$\mathrm{TM}$接受的语言设$\omega\in\Sigma^{*},\sigma_{0}=q_{0}\omega$称为关于输人$\omega$的初始格局。如果$M$从初始格局$\sigma_{0}=q_{0}\omega$开始的计算结束在接受的停机格局,则称$M$接受字符串$\omega$。$M$接受的字符串全体称为$M$接受的语言,或$M$识别的语言,记作$L(M)$。即
递归可枚举语言(r.e.语言)图灵机接受的语言称为递归可枚举语言。
{5.生要定理}
定理$10.1L$是0型语言当且仅当$L$是r.e.语言,换句话说,$L$由文法生成当且仅当$L$被TM接受。
数学家和计算机科学家们普遍接受下述看法。
丘奇(Church)论题人们所说的可计算的概念就是指TM可计算的。
定理10.2对于$i=2,1,0$,每个$i+1$型语言都是$i$型语言,并且这个包含关系是真的,即存在非$i+1$型的$i$型语言。
定理10.3设语言$L$,下述命题是等价的。
(1)$L$由右线性文法生成。
(2)$L$由左线性文法生成。
(6)$L$用正则表达式表示。
本章介绍了形式语言的基本概念,正则文法与有穷自动机的概念和基本性质,以及图灵机的基本概念。图灵机是最基本的计算模型之一。形式语言与自动机是计算理论的重要内容,特别是正则语言与上下文无关语言在编译理论中扮演着重要角色。此外,有穷自动机还被广泛应用于自动装置的电路设计中。
